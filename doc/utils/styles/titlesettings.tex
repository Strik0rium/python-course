\usepackage{titlesec} % 控制章节标题样式
\usepackage{tocloft} % 控制目录样式
\usepackage{xcolor}  % 自定义颜色

\titleformat{\chapter}[display]
    {\LARGE\bfseries\filleft}
    {\setlength{\parskip}{0em}\rule{\textwidth}{0.4pt} \\ 第 \thechapter 章}{0pt}{\setlength{\parskip}{0em}\Huge}
\titleformat{\section}[hang]{\LARGE\bfseries}{\thesection}{1em}{}
\titleformat{\subsection}[hang]{\Large\bfseries}{\thesubsection}{1em}{}
\setcounter{tocdepth}{2}
\setcounter{secnumdepth}{3} % 设置编号深度,启用 subsubsection 的编号
\renewcommand{\thesubsubsection}{\arabic{subsubsection}.}
\titleformat{\subsubsection}[hang]{\large\bfseries}{\thesubsubsection}{0.5em}{}

% 设置标题上下间距
% 参数分别为:左缩进、段前间距、段后间距
\titlespacing*{\section}{0pt}{1em}{0em} % section 标题与正文的距离
\titlespacing*{\subsection}{0pt}{1em}{0pt} % subsection 标题与正文的距离
\titlespacing*{\subsubsection}{0pt}{1em}{0pt} % subsubsection 标题与正文的距离
\titlespacing*{\paragraph}{0em}{0em}{1em} % 段落标题与正文的距离

% 设置段落缩进和段间距
\setlength{\parindent}{0em} % 段首不缩进
\setlength{\parskip}{1em} % 段落之间增加 0.5em 的间距

\cftsetindents{section}{1em}{2em}
\cftsetindents{subsection}{2em}{3em}