% 配置 Menlo 字体
\usepackage{fontspec}
\setmonofont{JetBrains Mono}

% 支持代码块显示
\usepackage{fancyvrb}
\usepackage{listings}
\usepackage{xcolor}

% 通用配置
\lstset{
    basicstyle=\ttfamily\small\heiti, % 设置代码基本样式
    % numbers=left,                    % 在左侧显示行号
    % numberstyle=\small\fontspec{Menlo}, % 使用 Menlo 字体显示行号
    % stepnumber=5,                    % 行号步长
    % numbersep=8pt,                   % 行号与代码的距离
    % frame=leftline,                  % 左侧显示边框线
    % framextopmargin=0em,
    % aboveskip=0.5em,
    % framexbottommargin=0em,
    % belowskip=-0.5em,
    xleftmargin=2em,                 % 代码块的左边界缩进
    xrightmargin=2em,
    % framexleftmargin=0pt,            % 边框的额外左边距
    breaklines=true,                 % 代码超出行自动换行
    showstringspaces=false,          % 禁止在字符串中显示空格
    % extendedchars=false,             % 避免章节标题等汉字显示问题
    % escapeinside={(*}{*)},           % 支持在代码块中短暂逃逸到 LaTeX 环境
    keepspaces=true,
}

\lstdefinelanguage{CFG}{
  morekeywords={
    if, else, abstract, assert, boolean, break, byte, case, catch,
    char, class, const, continue, default, do, double, enum, extends,
    final, finally, float, for, goto, implements, import, instanceof,
    int, interface, long, native, new, package, private, protected,
    public, return, short, static, strictfp, super, switch, synchronized,
    this, throw, throws, transient, try, void, volatile, while,
    id, relop, number, letter, digit
  },
  alsoletter={->},
  keywordstyle=\bfseries,
}

\renewcommand{\lstlistingname}{代码块}  % 改掉默认的 Listing
\DeclareCaptionFormat{mylst}{\hrule #1. #2#3}
\captionsetup[lstlisting]{format=mylst,labelfont=bf,singlelinecheck=off,labelsep=space}