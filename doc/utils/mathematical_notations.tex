\section*{数学符号}

本书采用的数学符号与 \LaTeX 标准库与 \lstinline|amsmath| 包中定义的数学函数基本一致.

\subsection*{逻辑符号}

\begin{itemize}
    \item $\forall$ 表示{\kaishu 对于任意的}.
    \item $\exists$ 表示{\kaishu 存在}.
    \item $\exists !$ 表示{\kaishu 存在唯一的}.
    \item $p$, $\bar p = \lnot p$ 表示命题 $p$ 及其否定 $\lnot p$,
    或者布尔变量 $p$ 及其布尔补 $\bar p$,
    它们是命题逻辑表达式 (布尔代数表达式) 中的{\bfseries 字面量} (literal).
    在下文中, 我们通常不加以区分布尔变量和命题.
    \item $pq$ 表示布尔变量 $p$ 与 $q$ 的{\bfseries 合取},
    $p+q$ 表示布尔变量 $p$ 与 $q$ 的{\bfseries 析取}.
    \item $p \to q$ 表示命题 $p$ {\bfseries 蕴含} (implies) $q$.
    \item $p \leftrightarrow q$ 表示 $p$ 与 $q$ {\kaishu 具有相同的真值}.
    在分析学中, 为了避免与极限趋近 $x \to x_0$ 混淆,
    蕴含记作 $p \Rightarrow q$,
    具有相同的真值记作 $p \Leftrightarrow q$.
\end{itemize}

\subsection*{集合}

\begin{itemize}
    \item 大写字母 $A, B, C$ 表示集合.
    \item 小写字母 $a, b, c$ 表示集合中的元素.
    \item $\varnothing$ 表示空集.
    \item $\Omega$ 表示论域或者样本空间.
    \item $\mathcal{P}(\Omega)$ 表示集合 $\Omega$ 的幂集.
    \item $|A|$ 表示集合 $A$ 的阶, 即 $A$ 中元素的个数.
\end{itemize}

\subsection*{常数}



\begin{itemize}
    \item $1$, T 和 \lstinline|true| 表示逻辑真, 或者算术中的数字 $1$.
    \item $0$, F 和 \lstinline|false| 表示逻辑假, 或者算术中的数字 $0$.
    \item $\mathrm{i}$ 表示虚单位, 它满足 $\mathrm{i}^2 = -1$, 在工程学中记作 $\mathrm{j}$.
    \item $\pi$ 表示圆周率.
    \item $\mathrm{e} := \lim_{n \to \infty} \left[ 1 + \dfrac{1}{n} \right]^n$.
    \item $\ivec, \jvec, \kvec$ 分别表示 $x, y, z$ 坐标向量
    $[1, 0, 0]^\mathrm{T}$, $[0, 1, 0]^\mathrm{T}$ 和 $[0, 0, 1]^\mathrm{T}$.
\end{itemize}

\subsection*{标准数学函数}



\begin{itemize}
    \item $\sin$, $\cos$, $\tan$, $\cot$, $\sec$, $\csc$ 表示三角函数.
    \item $\arcsin$, $\arccos$, $\arctan$ 表示反三角函数.
    \item $\sinh$, $\cosh$ 表示双曲函数.
    \item $\exp(x)$ 表示指数函数 $\mathrm{e}^{x}$, 这在指数部分较为复杂时
    具有更好的视觉效果, 例如 $\exp\left[ 1 + \dfrac{1}{x} \right]$,
    或者 $\exp(u \mathrm{e}^v) = \mathrm{e}^{u\mathrm{e}^v}$.
    \item $\log$, $\ln$, $\lg$ 表示对数函数.
    \item $\lim$ 表示极限.
    \item $\ker$ 表示映射的核或者矩阵的零空间.
    \item $\min$ 表示最小值, $\max$ 表示最大值.
    \item $\inf$ 表示下确界, $\sup$ 表示上确界.
    \item $\liminf$ 表示下极限, $\limsup$ 表示上极限.
    \item $\deg$ 表示图中节点的度, 以及多项式的次数.
    其中, $\deg^-(v)$ 表示节点的入度, 即以节点 $v$ 为终点的边的个数,
    $\deg^+(v)$ 为节点的出度, 即以节点 $v$ 为起点的边的个数.
    \item $\det$ 表示方阵的行列式.
    \item $\rank$ 表示矩阵的秩.
    \item $\dim$ 表示线性空间的维度.
    \item $\Pr(A)$ 表示随机事件 $A$ 发生的概率.
    \item $\hom(\mathcal{U}, \mathcal{W})$ 表示线性空间 $\mathcal{U}$ 到
    $\mathcal{V}$ 的所有线性映射组成的集合.
    \item $\arg$ 表示参数或者复数的幅角. 例如, $\argmin_{x \in X} f(x)$ 是 $X$ 中使目标函数取最小值
    的参数 $x$, $\arg z$ 是复数 $z$ 的幅角.
    \item $\Res_{z = z_0} f(z)$ 表示复变函数 $f(z)$ 
    在极点 $z = z_0$ 处的留数.
    \item $\floor{x}$ 表示不大于 $x$ 的最大整数.
    \item $\ceil{x}$ 表示不小于 $x$ 的最小整数.
    \item $\round{x}$ 表示 $x$ 四舍五入到整数的结果.
    \item $\sinc t$ 遵从 IEEE 和 ISO 80000-2:2019 的标准, 定义为 $\sinc t = \dfrac{\sin \pi t}{\pi t}$, 同时定义 $\sinc 0 = 1$.
\end{itemize}

\subsection*{关系}

\begin{itemize}
    \item $=$ 表示相等.
    \item $:=$ 表示定义, 即 $a := b$. 表示 “{\kaishu $a$ 据定义等于 $b$}”.
    \item $\equiv$ 表示等价或者恒等. 在某个区间 $[a, b]$ 内,
    $f(x) \equiv g(x)$ 事实上说明 $\forall x \in [a, b] \left(
        f(x) = g(x)
    \right)$.
    \item $\preceq$ 用来表示一般的偏序关系.
    \item $\sim$ 用来表示一般的等价关系.
    \item $\bar a$ 表示在等价关系 $\sim$ 下, 由元素 $a$ 确定的等价类.
\end{itemize}

\subsection*{可迭代对象、求和与连乘}


\begin{itemize}
    \item $1..n$ 表示一个可迭代对象, $k = 1..n$ 就是让 $k$ 遍历
    $1$ 到 $k$ 的所有整数, 如果是可列无穷多个, 则记作 $1..*$.
    这种记法在 Swift 和 Kotlin 这样的高级程序设计语言当中是常见的.
    \item 我们把 $\sum_{i=1}^n$ 也记作 $\sum_{i=1..n}$,
    这样的话我在打字的时候会更省力一些.
\end{itemize}

\subsection*{向量空间与矩阵}



\begin{itemize}
    \item $\mathbb{F}^m$ 表示 $m$ 维向量空间, 它的元素是 $\mathbb{F}$ 中的 
    $m$ 个元素构成的 $m$ 元有序数组.
    \item $\mathrm{M}_{m \times n}(\mathbb{F}) = \mathbb{F}^{m \times n}$
    表示域 $\mathbb{F}$ 上的 $m \times n$ 矩阵全体.
    \item $\mathrm{M}_n(\mathbb{F}) = \mathbb{F}^{n \times n}$ 
    表示域 $\mathbb{F}$ 上的 $n$ 级矩阵全体.
    \item $\mathrm{GL}_n(\mathbb{F})$ 表示域 $\mathbb{F}$ 上的 $n$ 级可逆矩阵的全体,
    我们也把这个集合称为{\bfseries 一般线性群} (general linear group).
    \item $\mathcal{V}$ 表示 $\mathbb{F}$-线性空间.
    \item $(u, v)$ 或者 $u \cdot v$ 表示 $u$ 与 $v$ 的内积.
    \item $\|v\|$ 表示 $v$ 的范数.
    \item $\dist_X(u, v)$ 或者 $d_X(u, v)$ 
    表示度量空间 $X$ 中 $u$ 与 $v$ 的距离.
    \item $\mathbf{A}$ 表示矩阵.
    \item $\vec{v}$ 表示向量.
    \item $a_{ij}$ 和 $\mathbf{A}[i; j]$ 表示矩阵中的元素.
    \item 我们用类似 Excel, MATLAB 和 NumPy 的表示方法来规定子矩阵的
    符号表示. 
    \begin{itemize}
        \item 规定
        \[ \mathbf{A}[i_1, i_2; j_1, j_2] := \begin{bmatrix}
            a_{i_1j_1} & a_{a_1j_2} \\ 
            a_{i_2j_1} & a_{a_2j_2}
        \end{bmatrix}, \]
        这里使用逗号连接两个行标和列表表示仅选择这些行和列交叉处的元素.
        \item 规定 \[ \mathbf{A}[i_1:i_2; j_1: j_2]
        := \begin{bmatrix}
            a_{i_1j_1} & \cdots & a_{a_1j_2} \\ 
            \vdots & \ddots & \vdots \\ 
            a_{i_2j_1} & \cdots & a_{a_2j_2}
        \end{bmatrix}, \]
        这里使用冒号连接两个行标和列表表选择这些行和列之间所有行和列交叉处的元素.
    \end{itemize} 
\end{itemize}

\subsection*{信号与系统}



\begin{itemize}
    \item $r_\zi$ 表示系统的零输入响应.
    \item $r_\zs$ 表示系统的零状态响应.
\end{itemize}
