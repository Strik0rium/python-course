\section*{排版约定}

\subsection*{为什么没有首行缩进?你是不是不专业?}

正如标题写的这个问题所说, 本书没有按照传统的排版要求和规范来设置首行缩进, 
其实只要留意观察你就会发现, 像 O'Reilly Media 的技术类书籍, 
还有一些其他的现代出版物, 都是这么做的. 
我想大概是有两个原因:
\begin{itemize}
    \item 如果在正文中频繁穿插文本、图表、公式、代码块, 以及数学上比较常见的
    各种各样的环境——举例 (example)、定理 (theorem)、证明 (proof)、命题 (proposition)、
    推论 (corollary)、公理 (axiom)——时, 仅占一行且有首行缩进的文本是不好看的.
    \item 越来越多的人选择使用手机或者平板电脑来阅读现代出版物, 
    这就导致水平方向上的版面空间是更紧张的, 而竖直方向则几乎拥有无限的延展度. 
    设计学总是会朝着适应媒介的方向去演化的, 因此取消首行缩进也就逐渐地普遍起来了. 
\end{itemize}

\subsection*{关于标点符号}

细心的读者可能发现文档中使用的是西文标点符号, 这是因为笔者本身是理工类专业的本科生,
在平时的文档编辑中, 为了防止中文句号 “。” 与一些数学符号例如 $\circ$ 混淆,
统一使用西文标点符号进行排版 (哪怕目标文档是一份人文类学科的文档).

\subsection*{排版约定}

我们在文档中遵循如下的排版约定:
\begin{itemize}
    \item {\bfseries 黑体} 表示新术语. 
    \item {\kaishu 楷体} 表示一些需要强调的内容, 或者是引用了别人的、我自己的话. 
    \item 等宽字体 \lstinline|constant width| 表示程序片段, 以及正文中出现的
    变量、函数名、数据库、数据类型、环境变量、语句和关键字等. 
\end{itemize}

% 此外, 对于一些需要强调、重点说明的信息, 以及一些容易出问题的关键环节, 我们会通过框体的形式来加以强调:
% \begin{center}
%     \begin{minipage}{0.3\textwidth}
%         \begin{hintbox}[frametitle={提示或建议}]
%             淡蓝色的框体表示提示或建议、笔者的评价以及复习备考的指南. 
%         \end{hintbox}
%     \end{minipage}
%     \hspace{1em}
%     \begin{minipage}{0.3\textwidth}
%         \begin{notebox}[frametitle={注记}]
%             淡绿色的框体表示一般注记、一些拓展性或补充性的内容. 
%         \end{notebox}
%     \end{minipage}
%     \hspace{1em}
%     \begin{minipage}{0.3\textwidth}
%         \begin{warningbox}[frametitle={警告或警示}]
%             淡粉色的框体表示警告或警示, 以及课程重点, 这些内容需要认真对待. 
%         \end{warningbox}
%     \end{minipage}
% \end{center}

